\section{Conclusion}

We have evaluated several compressed data structures in respect of their random access and substring queries and also their space consumption.
When it comes to query speed, \emph{block trees} show the highest query speed, beating all others on any dataset by various amounts.
On the other hand, if compression is the priority, then grammars show compression matching or beating block trees in many cases except on extremely repetetive data.
Block trees were only keeping up by not storing any prefix/suffix data (A2PS0) and while compressing well, as seen in \cref{fig:03:ssspeedbt} have a very slow substring query speed.

When observing tradeoffs, block trees of high arity and with high amounts of prefix/suffix data show the highest query speed while grammars offer high compression albeit at slower query speeds.

In conclusion, if on repetetive data high compression is the most important objective and substring queries are of lower importance,
then lower arity block trees without any prefix/suffix data seem to offer the best trade-offs.
They compress reasonably well in most cases and also offer fast random access queries.

If substring queries are a priority then high-arity block trees with prefix/suffix data or sampled scan seem to be the best choice.
The block trees seem to compress a less efficiently in the majority of cases but offered higher performance.
Grammars on the other hand seemed to compress a bit more consistently and efficiently, while having worse performance.
